%%%%%%%%%%%%%%%%%%%%%
\chapter{Introdução}%
%%%%%%%%%%%%%%%%%%%%%

%Primeira Parte - Breve Historico e Revisão Bibliográfica 
A crescente quantidade de informação na \textit{Web} nos últimos anos, fez das ferramentas de busca algo indispensável na coleta de informação. O algoritmo \textit{PageRank} é a técnica chave do Google, proposta inicialmente em \cite{brin2012reprint} no ano de 1998. O Google é um dos mais bem sucedidos buscadores e o \textit{PageRank} é com certeza um dos fatores deste sucesso.

É possível encontrar na literatura artigos recentes \cite{lei2015distributed,ishii2014pagerank} que ilustram o crescente interesse da comunidade de sistemas e controle no tema. Tal interesse advém das diversas dificuldades que são encontradas no cálculo do \textit{PageRank}, devido a sua complexidade. O problema é de grande dimensão e em geral trata-se de um sistema com variações abruptas em sua estrutura. Além disso, a variabilidade da \textit{Web} torna necessária a atualização frequente do cálculo. Ademais, a possibilidade do cálculo ser efetuado através de recursos computacionais distribuídos \cite{ishiiTAC12} evidencia o caráter desafiador deste problema.

Para o cálculo do \textit{PageRank} algumas considerações devem ser feitas e um fator crítico a ser levado em conta é o tamanho da estrutura da \textit{Web}. A \textit{Web} até 2010 era composta por pouco mais de 8 bilhões de páginas \cite{ishiiTAC10}, lembrando ainda que esse número está em constante crescimento. Atualmente a estimação do \textit{Ranking} das páginas é feita de forma centralizada no Google, onde toda a coleta de dados da \textit{Web} é feita por rastreadores, ou \textit{Crawlers}, que a acessam de forma autônoma.

A estrutura da \textit{Web} pode ser descrita como um grafo ${\cal G} = (\nu,\epsilon)$ \cite{ishiiSCL12}, sendo cada página tratada como um vértice $\nu = \lbrace 1, 2, ..., n \rbrace$ e cada \textit{link} como uma aresta dada por $\epsilon \subseteq \nu \times \nu$. Assim, se uma página $i$ possui um \textit{link} para $j$, então $(i,j) \in \epsilon$. O algoritmo do \textit{PageRank} também está associado a conceitos da área de sistemas e controle. Seu cálculo dá-se por uma equação de diferença \cite{pagerankSIREV}, cuja a matriz de transição é do tipo estocástica, devido a possibilidade do conjunto de páginas da \textit{Internet} poderem ser modeladas como uma cadeia de Markov de estados discreto \cite{costafragosomarques}. 

%Segunda Parte - Objetivo Geral e Objetivo Específico
Este trabalho trata-se de uma revisão dos modelos propostos para o algoritmo do \textit{PageRank} em \cite{ishii2014pagerank}. Assim, inicialmente foi necessário um estudo dos conceitos por trás dos modelos matemáticos do algoritmo, para que em seguida pudessem ser realizadas as simulações e serem feitas conclusões a partir delas. Foram realizadas simulações desde os modelos mais simples a aqueles que em um primeiro momento apresentaram problemas nas simulações. Por fim, são analisados os resultados e comparados os modelos e recursos usados nas simulações.

%Terceira Parte - Detalhamento das Secções
Assim, serão abordados conceitos de sistemas estocásticos no Capítulo II. Um histórico e detalhes sobre o \textit{PageRank} são apresentados no Capítulo III. Os modelos que regem o algoritmo são abordados no Capítulo IV. Os resultados das simulações dos modelos estão relatados no Capítulo V. Por fim, o trabalho é concluído e são propostas futuras pesquisas e implementações no tema.