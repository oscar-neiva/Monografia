%%%%%%%%%%%%%%%%%%%%%%%%%%%%%%%
\chapter{Considerações Finais}%
%%%%%%%%%%%%%%%%%%%%%%%%%%%%%%%

%Retornar o assunto principal.
Desde o início das pesquisas em desenvolvimento de ferramentas de busca em 1994, conseguir fazer um ranqueamento de páginas \textit{Web} de acordo com os interesses do usuário tem sido a parte mais desafiadora no desenvolvimento desses sistemas. Foi talvez por partir desse problema, e com uma boa solução, que o buscador Google obteve tamanho sucesso desde sua criação ao final da década de 90. As técnicas e modelos por trás do ranqueamento desse conjunto de documentos foram publicadas e, desde então, a comunidade científica tem trabalhado na pesquisa de novos modelos para tornar o cálculo do \textit{PageRank} mais eficiente.

O cálculo do \textit{PageRank} pode ser representado por diversos modelos, mas de forma geral é apresentado como uma equação de diferença, com algumas modificações na matriz de transição, a fim de tratar possíveis problemas de convergência. Os modelos e as propostas mais recentes do algoritmo estão voltados para o problema da distribuição do cálculo do ranquemento, no intuito de torná-lo cada vez mais eficiente. Mas a atenção ao algoritmo não tem por fim somente o ranqueamento de páginas \textit{Web}, a ideia do cálculo do \textit{PageRank} pode ser aplicada a diversos outros problemas.

%O que conclui-se do trabalho, objetivo geral e contribuições. Porque o meu trabalho é importante?
Durante o trabalho foi realizada uma revisão dos modelos propostos para o algoritmo do \textit{PageRank} na literatura. De forma que, inicialmente fosse efetuado um estudo dos modelos matemáticos por trás do algoritmo e em seguida fossem realizadas as simulações. Também para que fossem compreendidos os modelos matemáticos por trás do algoritmo, foi realizado um estudo sobre questões relacionadas a sistemas dinâmicos, sistemas estocásticos e probabilidade. %Pode-se então adquirir bastante conhecimento com a revisão desses assuntos e com as análises feitas a partir dos resultados. 

%Indicar trabalhos futuros a serem desenvolvidos.
Neste trabalho ainda pretende-se utilizar outros métodos válidos na simulação do \textit{PageRank}. Como a cada ano novos artigos são publicados no tema, novos modelos e técnicas estão sempre sendo publicados pela comunidade científica. Porém uma atenção maior seria dada a modelos agregados de cadeias de Markov, e problemas de consenso. E ainda como uma continuação deste trabalho, poderia ser feita uma implementação com \textit{links} já coletados por um \textit{Web Crawling}. De forma a por em prática os modelos num ambiente real e usando-se de técnicas de computação distribuída, afim de otimizar o cálculo e por em prática os algoritmos distribuídos. Além de usar outras linguagens de programação, como C e C++, por questões de desempenho e por serem mais adequadas a implementações em \textit{clusters}.